
\section{Introduction}
%formation protoplanetary systems
%formation debris disks
A protoplanetary system's evolution starts with the collapse of a molecular cloud, where its gravitational force overcomes the internal pressure. 
%
Various triggers are causing this collapse, such as local instabilities and supernova shocks \citep{Shu_1987}.
%
The collapse drives the pressure and temperature up in its centre, forming a protostar. 
%
Furthermore, the rotation velocity increases as the radius of the cloud decreases, thereby increasing the centripetal force in the direction of rotation due to conservation of angular momentum, and consequently flattening the cloud into a disk \citep{Williams_2011}. 
%
A protoplanetary system has formed \citep{Walker_1994,Yorke_1993}. 
%Evolution of stars within
The protostar accretes material from the disk, increasing its density and temperature \citep{Girichidis_2020}, reaching the criteria to start hydrogen fusion, thereby slowly evolving into a main-sequence star.
%
Simultaneously, the protoplanetary disk evolves into a planetary disk by accreting its dust onto the star in addition to dust dissipation by stellar winds and photo-evaporation \citep{Bai_2016}. 
%
Furthermore, collisions within the disk can lead to planetesimals, which will either accrete material with the help of gas interactions or fragment \citep{Wyatt_2008, Dominik_2006}.
%
Once planetesimals have formed, some successfully evolve into planets by accreting material from within their circumplanetary disk.

%
%formation CPD, rings and moons
A circumplanetary disk (CPD) forms around a planet \citep{Taylor_2024} within its Hill sphere, defined as the radius where dust is gravitationally bound to its planet. 
%
In addition to dust and gas within the CPD being accreted onto the planets, some slowly dissipated away, while others collide and fragment \citep{Oberg_2020}.
%
The collisions can lead to dust clumping and eventually exosatellites spawning. 
%
These exosatellites can furthermore clear their path, creating gaps and rings as seen for the rings around Saturn, seen as photometric variations during their Hill sphere transit \citep{Canup_2002}.
%
In this paper we will focus on the Hill sphere transit of \pic c.
%

%\section{The \pic system}
The enthusiasm for the \pic system began to grow after the discovery of its edge on debris disk in the 80's \citep{Smith_1984, van_leeuwen_2007}.
%
It was one of the few to be resolved due to its close proximity at 19.4 pc, dominating the research field of debris disks \citep{Artymowicz_1997} with an inner warped disk at 5\textdegree \ from the main disk \citep{Golimowski_2006}.
%
The system is also relatively young at 21 million years old, hosting an A6 star.
%
The primordial dust from the interstellar molecular cloud has likely been dispersed, leaving a system composed of disintegrating bodies in an early phase planetary system \citep[e.g.][]{Backman_1993,Artymowicz_1997,Lagrange_2000,Zuckerman_2001}.
%
Furthermore, star-grazing comets have provided evidence of their collisions and evaporation, giving rise to a mix of gas and dust in the disk  \citep[e.g.][and references therein]{Lagrange_1998,Beust_1990,Lecavelier_Des_Etangs_1995,Beust_1996,Vidal_Madjar_2017}. 

%beta pic b
The structure and composition of the disk led to the belief that the system hosts exoplanets and indeed, in 2009, the first exoplanet of the system was detected by means of high contrast imaging  \citep{Lagrange_2009}. 
%
% A new technique called high contrast imaging reduces the star's signal by a couple of magnitudes to allow for the exoplanet flux to be detected. 
%
The exoplanet \pic b was found to have a mass of 11 $M_{Jup}$ and a period of 21 years orbiting at 9.8 au  \citep[e.g.][and many more]{Lagrange_2010, Chauvin_2012,Wang_2016,Snellen_2018,Kenworthy_2021}. 
%
%beta pic c
Many years later, \pic c was discovered through the radial velocity technique  \citep{Lagrange_2019,Lagrange_2020,Nowak_2020}, mapping red and blue shifted radial velocities of the star caused by its companion's gravitational pull. 
%
\cite{Lagrange_2020} found the exoplanets to be of comparable size with \pic b with a mass of 7.8 $\pm$ 0.4 $M_{\text{Jup}}$, however orbiting at closer proximity to the star with a semi-major axis of 2.7 $\pm$0.02 au. 
%
Due to its low contrast at close orbit, current instruments lack the capability of performing high contrast imaging to constrain the orbital parameters.
%
Therefore, the astrometry of \pic b was used to derive the astrometry of \pic c, giving an approximate orbital period of 3.3 years \citep{Lacour_2021}. 
%
Before the planets of the \pic system were discovered, a mysterious 1981 event occurred.



\begin{figure}[t]
 \centering
    \includegraphics[width=0.45\textwidth]{Figures/Photometry/m1981model.pdf}
    \caption{The data of the 1981 event with errorbars (red). The grey dots did not pass the quality check \cite[][]{Lamers_1997,Lecavelier_Des_Etangs_1995}. The model (blue) consists of a broad optically thin with a narrow optically thick ring at its centre. The magnitude of the event is 0.035 \citep{Lamers_1997}.  Credit: \citet{Kenworthy_2021}}
    \label{fig:1981}
\end{figure}

%\section{The 1981 event}
On the 10th of November 1981 the Geneva Observatory measured a significant increase in photometric flux of the \pic system \citep{Lecavelier_des_Etangetangs_1994, Lecavelier_Des_Etangs_1995}. 
%
This mysterious 1981 event shown in Figure \ref{fig:1981} lasted 10 days with a dip halfway through.
%
They detected the event in multiple color bands and found the statistical noise to be less than $10^{-5}$. 
% 
Thus, with a 99$\%$ confidence level, the origin of the event was highly hypothesised.
%
One cause could be a giant comet \citep{Lamers_1997}, causing a passing cloud with highly forward-peaked scattering. 
%
The other is a transiting planet \citep{Lecavelier_des_Etangetangs_1994,Lecavelier_Des_Etangs_1995, Lecavelier_Des_Etangs_1997}.
%
Constraints of a planet producing the 1981 event have been made assuming a circular orbit \citep{Lecavelier_Des_Etangs_1997}; 1) The period is less than 19 years; 2) the radius of the transiting object must be 2.3 to 4.0 times the radius of Jupiter.  
%
Lastly, the event could be caused by dust around an occulting planet with Rayleigh scattering, which would explain the photometric dip as a stellar limb-darkening effect.
%Unfortunately, the event has not been detected since.
%
The earlier motion that the 1981 event was caused by a planet has thus become less supported, as observations have shown that the large occultation depth is more likely to be produced by dust in a CPD \citep{Lecavelier_des_Etangs_2008a, Lecavelier_des_Etangs_2008b} and a similar transit to the 1981 event has been observed for the planet Fomalhaut b in a young debris disk \citep{Kalas_2008}. 
%
Fortunately, the hypothesis of a transiting CPD disk causing the 1981 event can be tested.
%
\pic b and c both have a high enough orbital inclination that they do not transit \citep{Wang_2016}.
%
However, their Hill spheres, possibly hosting a circumplanetary disk, do and photometric data of the Hill sphere transit of \pic b was fitted to the model of the 1981 event (figure \ref{fig:1981}) from \cite{Kenworthy_2021}.
%
Unfortunately, no compelling results were found \citep{Lecavelier_Des_Etangs_2009,Kenworthy_2021}. 
%
With the discovery of \pic c, the 1981 model from \cite{Kenworthy_2021} can be applied to \pic c to see if we can solve this mystery.


%, where \pic c has a similar orbital inclination as \pic b  \citep{Wang_2016}.
%




%\section{The circumplanetary disk}
%Cpd disk intro
%

The Hill sphere transit also gives insight into the circumplanetary disk (CPD). 
%
The CPD lies within the Hill sphere and can host rings and moons, as seen in Saturn and Jupiter. 
%
However, outside of our solar system, only candidates for rings and exosatellites have been detected \citep{Osborn_2019, Kenworthy_Mamajek_2015,Mamajek_2012,Rieder_2016}. 
%
PDS 110b is hypthesized to host rings as two eclipses within a short timespan has been observed \citep{Osborn_2019}. 
%Unfortunately the observations and models do not match. 
Similarly, the J1407b ring system underwent multiple complex eclipses, possibly due to a substellar companion hosting rings \citep{Kenworthy_Mamajek_2015,Mamajek_2012,Rieder_2016}. 
%
Unfortunately, in both cases confirmation of exomoons and rings has not been successful. 
%
Stimulation modules show detection is possible with more sensitive instruments \citep{Alshehhi_2020,Akinsanmi_2017}. 
%
For PDS 70c detection and confirmation of the CPD was possible \citep{Benisty_2021}. 

%Cpd disk pictoris
The \pic system is young and there is reason to believe its exoplanets hosts a CPD. 
%
\pic b has been probed during its transit in 2017 and 2018 \citep{Kenworthy_2021} to fit for possible tilts and inclinations of the CPD extended to multiple radii and it's mass limits. 
%
Unfortunately, no compelling evidence was found. 
%
This study follows the methodology outlined by \cite{Kenworthy_2021}, however applied to \pic c. 











