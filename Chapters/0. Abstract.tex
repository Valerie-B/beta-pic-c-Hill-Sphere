\abstract{\pic is a young and nearby planetary system hosting an edge-on debris disk and multiple planets. The young planets could host a circumplanetary environment with exomoons and rings. Furthermore, a mysterious photometric dip was observed in 1981, possibly due to the presence of an exoplanet. }
{A search for the origin of the mysterious 1981 event by analysing the transits of \pic c and additional discontinuous coverage from 03-02-2017 to 17-01-2023. Furthermore, probing the orientation and upper mass limit of a possible circumplanetary disk around \pic c.}
{Observations of BRITE, bRING and ASTEP were fitted to a model of the 1981 event to search for a similar detection. The sensitivity and analysis of a circumplanetary disk model following the methodology from \citet{Kenworthy_2021}. }
{No compelling evidence for the 1981 event was found between 28-11-2016 (MJD 57720) and 06-01-2023 (MJD 59950). The circumplanetary disk model was found to be limited to transits with an uncertainty of the midpoint of less than 3 days. Therefore, the transits at $\text{58210} \substack{+\text{28} \\ -\text{42}}$ and $\text{59413} \substack{+\text{10} \\ -\text{11}}$ yielded inconclusive disk orientations and upper mass limits.}
{The Hill sphere transit of \pic c lead to no compelling evidence to suggest it was the 1981 event, nor was there enough coverage to implement the CPD model with certainty. Furthermore, ill-constrained midpoints of the transits possess large constraints on the reliability of the results.} 
\keywords{planets and satellites: rings – planets and satellites: formation – stars: individual: $\beta \ Pictoris$ }

