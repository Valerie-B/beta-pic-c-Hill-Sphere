


\section{Appendix}\label{sec:appendix}
The upper mass limits shown in figure \ref{fig:taumass_58210} and \ref{fig:taumass_59413} are computed using equation \ref{eq:mass_limit} and the optical depths of figure \ref{fig:58210diskfit_3datasets_taumass} and \ref{fig:59413diskfit_3datasets_taumass}.
%
% Figure \ref{fig:taumass_58210} and \ref{fig:taumass_59413} in the Appendix \ref{sec:appendix} show the upper mass limit for MJD 58210 and 59413, respectively, using \ref{eq:mass_limit}.
%
% Table \ref{tab:uppermass} shows the mean dust size and upper mass limit constrained by the combination of the upper limits of $\tau$ among the different instruments.
%
% Thus, the upper mass limit at MJD 58210 was computed for inclinations less than 60\textdegree and tilts larger than 60\textdegree, resulting in a dust grain size of 18.1 and 25.6 $\mu$m, for 0.3 and 0.6 $r_{\text{Hill}}$ respectively. 
%
Figure \ref{fig:taumass_58210} shows upper limit of $\tau$ corrected for inclination in the top panel and the upper mass limit in the lower panel. 
%
The left panels have disks of 0.3 $r_{\text{Hill}}$ and the right of 0.6 $r_{\text{Hill}}$. 
%
The upper limit of $\tau$ at MJD 58210 is constrained to inclinations less than 60\textdegree and tilts larger than 60\textdegree and therefore the same constraints hold for the upper mass limit. 
%
The CPD of 0.6 $r_{\text{Hill}}$ is slightly more constrained with a higher grain size. 
% The upper $\tau$ limit does correspond to low SNR values as earlier mentioned.  
%
Figure \ref{fig:taumass_59413} shows the upper mass limit at MJD 59413, constrained using only bRing.
%
As a result, the upper limit of $\tau$ is less well constrained, allowing all possible orientations of the disk.  
%
There is a bias towards higher inclinations, agreeing with earlier motions.
%
The two different transits do not agree with each other, however many parameters are not well constrained. 



\begin{figure*}[b]
\centering
\begin{subfigure}{.5\textwidth}
  \centering
  \includegraphics[width=\linewidth]{Figures/Photometry/58210.0diskfit_taumass_030.pdf}
  %\caption{}
  \label{fig:sub1}
\end{subfigure}%
\begin{subfigure}{.5\textwidth}
  \centering
  \includegraphics[width=\linewidth]{Figures/Photometry/58210.0diskfit_taumass_060.pdf}
  %\caption{}
  \label{fig:sub2}
\end{subfigure}
\caption{The upper mass limit corrected for inclination (top panels) and the upper mass limit (lower panels) for MJD 58210 at 0.3$r_{\text{Hill}}$ (left panels) and 0.6$r_{\text{Hill}}$ (right panels). The optical depth results from  ASTEP, bRing and BRITE are combined to constrain the disk to less than 30\textdegree \ and tilts larger than 60\textdegree for 0.3$r_{\text{Hill}}$. At 0.6$r_{\text{Hill}}$ the orientation is more constrained. The grain sizes are 18.1$\mu$m and 25.6$\mu$m, with upper mass limits of $3.41x10^{20}$ and $9.15x10^{20}$ g, respectively.  }
\label{fig:taumass_58210}
\end{figure*}  
%
\begin{figure*}[b]
\centering
\begin{subfigure}{.5\textwidth}
  \centering
  \includegraphics[width=\linewidth]{Figures/Photometry/59413.0diskfit_taumass_060.pdf}
  %\caption{}
  \label{fig:sub1}
\end{subfigure}%
\begin{subfigure}{.5\textwidth}
  \centering
  \includegraphics[width=\linewidth]{Figures/Photometry/59413.0diskfit_taumass_060.pdf}
 % \caption{}
  \label{fig:sub2}
\end{subfigure}
\caption{The upper mass limit corrected for inclination (top panels) and the upper mass limit (lower panels) for MJD 59413 at 0.3$r_{\text{Hill}}$ (left panels) and 0.6$r_{\text{Hill}}$ (right panels). The upper optical depth is not constrained due to coverage only by bRing. There is a bias towards larger inclinations. The grain sizes are 25.6$\mu$m, with upper mass limits of $8.07x 10^{20}$ and $5.11x10^{21}$ g.}
\label{fig:taumass_59413}
\end{figure*} 


