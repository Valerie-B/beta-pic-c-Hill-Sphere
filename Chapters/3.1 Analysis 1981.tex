\section{Photometric results $\&$ analysis}\label{sec:Photo_Results}
\subsection{The 1981 event}\label{sec:method_1981}
\begin{figure*}[b] %31.3cm
    \centering
        \includegraphics[trim={0 67.cm 0 0},clip,width=\textwidth] {Figures/Photometry/1981_fits.pdf} 
         \caption{The amplitude $a$ of the BRITE (blue), bRing (green) and ASTEP (red) data fitted to the 1981 event are shown, where 0.035 marks the magnitude of the 1981 event. In light blue, the primary (MJD 58210 $\&$ 589413) and secondary (MJD 58707) Hll sphere transits of \pic c are marked. No compelling detection during the transits or outside is seen among all three datasets. Furthermore, bRing and ASTEP show unknown periodic pulsations.}
     \label{fig:cross_correlation}
\end{figure*}
The lightcurve of the 1981 event (figure \ref{fig:1981}) is modelled as two rings, where an optically thick but narrow ring lies within a broader, optically thin one, as described in \citet{Lamers_1997}. 
%
The forward scattering from the optically thin ring passing in front of the star is characterised by a Gaussian feature with an FWHM of 3.2 days and a magnitude of 0.035 \citep{Lamers_1997}.
%
Furthermore, the segment of the ring transiting is approximated as a straight line, as the radius of the ring is much larger than the diameter of the star.
%
The 1981 event is fitted to the photometric data following \citet{Kenworthy_2021} and the model $F_{1981}(t)$ shown in figure \ref{fig:1981} is fitted for amplitude $a$ and offset $b$ using, 
%
\begin{equation}
F(t) = aF_{1981}(t_{mid}) + b.
\end{equation} 
%

% The fitting function takes $t_{mid}$ and masks out any datapoints more than 8 days away from the timestamp.  
%
% A fit is made every quarter-day, except when not enough data points are present; then the fit is forfeited. Furthermore, the non-linear fitting varies $t_{mid}$ by half a day to improve the fit and uses different $a$'s from 0 to 5.  
%
The function returns the best fit $a, b$ and $\sigma_a$, while discarding amplitudes larger than 0.05.
%
% The magnitude of the 1981 event is 0.035, therefore similar amplitudes could indicate a similar event. \\
%
$F(t)$ masks out any datapoints more than 8 days away from the timestamp, fitting only for the duration of the 1981 event.
%
Unfortunately, this does not take the uncertainties of the midpoints of the transits into account.
%
In addition, the 1981 event could occur beyond the transits due to unknown sources.
%
Therefore, all observational data from ASTEP, bRing and BRITE were fitted to the model, shown in figure \ref{fig:cross_correlation}.
%
bRing has the best coverage, while ASTEP's coverage is limited to the first 500 days, and BRITE's coverage is very sporadic.  
%
The transits of 1.5 months are marked in the blue box, while the uncertainties are marked by errorbars.
%
The magnitude of the 1981 event is indicated by a horizontal line ($a$=0.035). 
%
A good fit to the model would yield a smooth, symmetric increase and decline in amplitude to 0.035 during 8 days, simultaneously over multiple instruments.
%
The latter is unfortunately difficult to achieve due to a lack of coverage from ASTEP and BRITE. 
%
No such ideal fit is seen, however multiple timestamps exceed the 0.035 threshold.
%primary transit
The primary transit at MJD 58210 sees a symmetric curve at MJD 58250 beyond the detection threshold by ASTEP. 
%
Despite ASTEP showing a symmetric curve around the threshold of 0.035, BRING does not and BRITE does not have any coverage.
%
Therefore, the 1981 event likely did not occur during the first primary transit and could have been caused by systematic errors of ground-based instruments \citep{Kenworthy_2021}.
%
The second primary transit at MJD 59413 only has coverage from bRing, however there is a detection.
%
The errors among the amplitudes of the fit should be taken conservatively and therefore at MJD 58415 the detection threshold is just met.
%
Furthermore, the amplitudes around 0.01-0.02 have been seen (e.g. MJD 58815, MJD 59100, MJD 59760) and can not explicitly be caused by the 1981 event.
%
The rest of the detection ($a$>0.02) is short-lived for about a day or two, likely not the 1981 event.
%
The last transit is the secondary transit at MJD 58707, showing a detection (MJD 58690), even when taking the conservative amplitudes just outside the expected transit times.
%
This transit is characterised by backscattering, where the scattered light from within the Hill sphere of \pic c and the star interfere, resulting in a different measurement than when the starlight is blocked by dust within the Hill sphere.
%
Therefore, a different model is needed to characterise the light curve.
%
Thus, there is no strong evidence that the 1981 event was caused by the Hill sphere transit of \pic c.  
%Other detections

Figure \ref{fig:1981} shows a couple of extra features in addition to those during the transits. 
%
At MJD 57900, ASTEP observed an increase in flux, while BRTIE and bRING do not, possible due to systematic errors or snowstorms.
%
Furthermore ASTEP has extreme errors at MJD 58270 ASTEP and can not be trusted to be a valid detection.
%
At MJD 58070, MJD 58280, MJD 58750 and MJD 58930 bRing has a detection. 
%
In all cases either the errors are large or it does not align with the other instruments.
%
BRITE has the most accurate data as a space-based telescope, unfortunately, also the least amount of coverage and shorter observation times due to its orbits around Earth and no threshold detection at all.
%comparison pulsations
There are small pulsations with a magnitude increase of $a$ $\sim$ 0.01 over a span 5-15 days, which occur roughly periodically and sometimes overlap within multiple instruments (MJD 57960), however not always (MJD 57980).
%
Several mechanisms could create brightness pulsations in the \pic system, however many are shortlived.
%
The star has two pulsation modes, one around 30 and the other 40 minutes, with a magnitude smaller than 1.5mmag \citep{Koen_2003}.
%
Neither strong enough nor long enough.
%
Exocomets could also not cause such a feature. They are short-lived, typically lasting two days, are asymmetric and dim (<2mmag) \citep[e.g.][]{Heller_2024, Zieba_2019}.  
%
Passing clouds in the \pic system are not that abundant and last longer, similar to the Hill sphere transit of \pic b. 
%
BRITE is the only instrument with no pulsations, therefore giving reason for ground-based sources.
%
Neither satellites nor birds could explain these.
%
One possible explanation is starspots with a period of 4-5 days as seen in young clusters \citep{Cohen_2004, Oshagh_2013}.
%
There is much more to be researched about the cause of the pulsations and the 1981 event.


% Comparison with Previous Work
%
The search for the origin of the 1981 event has again been unsuccessful, however many more have been puzzled by this strange event \citep[e.g.][]{Lecavelier_Des_Etangs_1995, Lecavelier_Des_Etangs_1997, Kenworthy_2021, Lecavelier_Des_Etangs_2009, Lagrange_2019}).  
%
The hypotheses about a transiting planet or its Hill sphere have become more unlikely, as no compelling evidence was found with \pic b and \pic c \citep{Lecavelier_Des_Etangs_1997, Kenworthy_2021}.
%
% There was enough coverage to conclude that the event is unlikely to have happened during the \pic b transit.  
%
% The Hill sphere transit of \pic c leaves more uncertainty, however the primary transit at MJD 58210 has good coverage to conclude the transit is likely not linked to the 1981 event.
%
% Furthermore, all threshold detections outside the transits are not compelling.
%
Other transiting bodies could have created the 1981 event, however none large enough we are aware of.  
%
Exocomets are too short-lived nor bright enough.  
%
Another hypothesis is scattering by a dust cloud.  
%
Dust clouds are large, resulting in brightness variations at long timescales.  
%
However, at the right tilt and inclination, the dust cloud transit can be short enough to coincide with the 1981 event.  


%
Furthermore, large comets can create dust clouds, which either pass the line of sight at the perihelion or are fragmented and collide with the star \citep{Lamers_1997}, creating so called Shoemaker-Levi fragments.  
%
This comet is captured into a satellite orbit and eventually broken apart by tidal disruption events. 

% Consider alternative explanations for your findings.
In conclusion, the origin of the 1981 event remains unknown, however we see that the 1981 event is likely not correlated with any periods with coverage from BRITE, bRing and ASTEP.  
%
Therefore, also not the Hill sphere transit of \pic c.  
%
The 1981 model could be applied to a greater dataset to see if more detections exceed the 0.035 magnitude threshold.  
%
Furthermore, to study the periodic flux variations of bRing.  
%




\begin{figure*} %60.7cm
\ContinuedFloat
    \centering
        \includegraphics[trim={0 0 0 24.9cm},clip,width=\textwidth] {Figures/Photometry/1981_fits.pdf}
    \caption{See Figure above}
  \label{fig:cross_correlation2}
\end{figure*}