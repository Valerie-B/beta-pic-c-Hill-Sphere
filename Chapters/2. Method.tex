\section{Geometry of the Hill sphere 
transit}\label{sec:Hill_sphere_transit}

%
Radial velocity observations, imaging and GRAVITY observations \citep{Lacour_2020, GRAVITY_Collaboration_2021,Lagrange_2020} were combined to map the orbital parameters of \pic b and c by J. Wang \citep{Lacour_2021} using parameters from \citet{Vandal_2020}. 
%
The data was processed in the Python \verb|Orbitize| model \citep{orbitize} using 200 orbits as shown in figure \ref{fig:Orbitize_plot}.
%
The separation of \pic b and c from the host star is shown in Hill radii \citep{Hamilton_1992}, computed using,
%
\begin{equation}
    r_{\text{Hill}} \approx \frac{a}{d_{\star}} \left(\frac{M_p}{M_{\star}+M_p} \right)^{\frac{1}{3}}  (1-e),
\end{equation}
%r_hill_b_mas = 1000 * (rhill(M, Mb, ab)*(1-eb)).value / 19.44
Where the distance between the star and the planet $a$ is in au, M are the masses, $d_{\star}$ = 19.44 pc is the distance of the \pic system and $e$ is the eccentricity. 
%
The parameters of the star and \pic c are listed in
Table \ref{tab:orbitize_param} and are used to obtain a Hill radius of 0.30 au. 


\begin{figure}[t]
    \centering
    \includegraphics[width=0.5\textwidth]{Figures/Photometry/orbitize_b_c.pdf}
    \caption{The separation of \pic b and c from the star in terms of their Hill sphere.  \pic b has a longer orbital period ($\sim$ 20 yr), with a Hill sphere transit at MJD $\sim$58000 and a Hill sphere radius of 1.1 au. \pic c has a shorter orbital period ($\sim$ 3.3 yr), with two transits at MJD $58210^{+28}_{-42}$ and $59413^{+10}_{-11}$ and an eclipse at $58707^{+5}_{-8}$. The Hill sphere radius is 0.3 au. \pic b has been resolved by high contrast imaging, resulting in higher constraints of its orbital bundle.}
    \label{fig:Orbitize_plot}
\end{figure} 

\begin{table}
\centering
\begin{threeparttable}
\caption{Adopted observational values for the \pic System and \pic c}
\vspace{0.05in}
\label{bpicparams}
\begin{tabular}{|c |c |c| c|} %centre justified
\hline
Parameter\,&Value\,&Units\,&Reference\\
\hline
$M_*$       & 1.797 $\pm$ 0.035     & $M_\odot$ & 1\\
$R_*$       & 1.497 $\pm$ 0.025     & $R_\odot$ & 1\\
$T_*$       & 8090 $\pm$ 59         & K         & 1\\
$L_*$       & 8.47 $\pm$ 0.23     & $L_{\odot}$ & 1\\
$M_c$       & 8.5 $\pm$ 0.5    & $M_J$     & 4\\
$R_c$       & 1.2 $\pm$ 0.1     & $R_J$     & 3\\
$T_c$       & 1250 $\pm$ 50      & K         & 3\\
$P_c$       & 1221 $\pm$ 15      & days         & 2\\
  a           & 2.68 $\pm$ 0.02 & au       & 2 \\
  e          &   0.208 $\pm$ 0.0074   &       & 4 \\
\hline
\end{tabular}
\begin{tablenotes}
\small
\item (1) \citet{Zwintz_2019},
(2) \citet{Lacour_2021},
(3) \citet{Nowak_2020}. (4) \citet{Vandal_2020}
\end{tablenotes}
\label{tab:orbitize_param}
\end{threeparttable}

\end{table}

The transits occur at minimum separation, where the primary transit occurs as the Hill sphere passes before the star and the secondary as the Hill sphere transits behind the star. 
%
Figure \ref{fig:Orbitize_plot} shows one full orbit of \pic c, with two primary transits and one secondary transit. 
%
The most likely midpoint of the transits are set as the mode of the 200 orbits, with limits set by the earliest and latest timestamps of minimum separation. 
%
The computed primary transits are MJD $\text{58210} \substack{+\text{28} \\ -\text{42}}$ and $\text{59413} \substack{+\text{10} \\ -\text{11}}$. 
%
The secondary transit at MJD $\text{58707}\substack{+\text{5} \\ -\text{8}}$ has the smallest dispersion. 
%
This is due to observations constraining the orbit of \pic c \citep{Vandal_2020}. 
%
Orbits further away from MJD 58700 are less well constrained, leading to increasing uncertainties. 


%
The transits are expected to last $\sim$1.5 months \citep{Vandal_2020}, therefore the first primary transit at MJD 58210 has an error larger than the expected transit time. 
%
Furthermore, this transit has been observed with photometric data \citep{Zwintz_2019}, however no compelling photometric evidence was present. 
%
The latter primary transit at MJD 59413 coincides with the expected date from \citet{Vandal_2020}, however no observational data has yet to be analysed. 
%
The secondary transit does not probe backscattering and not the CPD
%



\section{Observations}\label{sec:Observations}

\begin{figure*}[t!]
    \centering
    \includegraphics[width=0.9\textwidth]{Figures/Photometry/All_photometric_transits.pdf}
    \caption{The binned photometric data of all Hill sphere transits of bRing, Brite and ASTEP. The latter two only cover the primary transit at MJD 56210. The midpoints of the transits and their uncertainty are marked with an errorbar and the blue zone marks the duration of the transit ($\sim$ 1.5 months). No significant decrease in flux is seen over the three transits. }
    \label{fig:photometric_transits}
\end{figure*} 

The Hill sphere campaign of \pic b \citep{Kenworthy_2021} observed the \pic system closely during the Hill sphere transit of \pic b 
%
Close thereafter, \pic c was discovered and the campaign happened to cover the transit of \pic c as well \citep{Stuik_2017}. 
%
One of the instruments used was \pic b Ring project (bRing) with almost full coverage of the system \citep{Snellen_2012,Snellen_2013,Stuik_2014, Stuik_2016,Talens_2017, Talens_2018}
%
This instrument's coverage was further supported by Mascara and DREAM from 03-02-2017 to 17-01-2023 and the reduced data was provided by R. Stuik.
%
The BRITE constellation (BRIght Target Explorer) \citep{Weiss_2014} observed \pic c from LEO, allowing higher precision than ground-based observations and negligible noise from atmospheric turbulence and scatterings.  
%
The data quality of the blue filter was deemed insufficient \citep{Zwintz_2019} for \pic. 
%
Therefore, only the red filter (550 -700 nm) was used. The data reduction pipeline can be found in \citet{Popowic_2017} and a correction for \pic in \citet{Zwintz_2019}. 
%
The data from BRITE-Heweliusz (BHr) and BRITE-Toronto (BTr) were calibrated and provided by K. Zwintz, covering  16-03-2015 to 28-03-2021 with irregular observations. 
%
The last instrument is ASTEP, a 40cm telescope located in Antarctica \citep{Crouzet_2010}, with an effective bandwidth from 575 to 760 nm. 
%
It's operational during the arctic winters and modified with a filter to observe the \pic system \cite{Kenworthy_2021}. 
%
Furthermore its background correction is suboptimal due to the high brightness of \pic relative to the instrument's specifications \citep{Mekarnia_2016}. 
%
In addition, snowstorms warp the incoming light due to the formation of ice and ice crystals and therefore need to be flagged out.
%
ASTEP obtained data from 30-03-2017 to 17-07-2018 over two winters and has the same data as provided by M. Kenworthy in \citet{Kenworthy_2021}.
%



% \section{Methods}\label{sec:Methods}
% The methodology explains in detail how the transit times of \pic c were determined. 
% %
% Then, the photometric instruments are used to observe the Hill sphere transit, a circumplanetary disk model that computes possible tilts and inclinations of the disk. 
% %
% Also included is a 1981 event model and its correlation with photometric data. 
% %

% All models and computations can provide insight into the Hill sphere of \pic c.

% \section{Photometric Analysis} \label{sec:Photo_Method}

The midpoints of the transits and their coverage are shown in figure \ref{fig:photometric_transits}. 
%
The data has been normalised and binned to 0.05 days.
%
The error-bars mark the uncertainty of the midpoint of the transits and the boxes mark the duration of the transit ($\sim$ 1.5 months). 
%
bRing covers all three transits, while ASTEP and BRITE only cover the first primary transit at MJD 58210 due to the \pic b Hill sphere transit campaign. 
%
Unfortunately, the first primary transit is not well constrained.
%
Furthermore, no significant stellar limb darkening effect is seen over the different instruments.
% 
The secondary transit at MJD 58707 observed backscattered light, while having the smallest constraints.
%
To probe the environment within the Hill sphere, a dimming of photometric flux due to dust within the Hill sphere is needed and not scattered light. 
%
Therefore, despite the small dip at MJD 58690 and the smallest uncertainty of the midpoint of the transit, the secondary transit does not give insight into the CPD.
%
The last transit at MJD 59413 has the least amount of photometric coverage and shows no significant flux deviations. 
%
Its midpoint is better constrained than the first primary transit.







